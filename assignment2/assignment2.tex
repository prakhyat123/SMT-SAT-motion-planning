\documentclass[a4paper, 12pt]{article}

\usepackage{escexam}

%\excludecomment{solution}

%\renewcommand*\ttdefault{cmvtt}

\begin{document}

\vspace*{14ex}

\makeheader{1}                              					% examination number (used to set theorem, lemma numbers)
           {March 15, 2020}      					         		% examination date or deadline
					 {40}											% total marks
					 {Homework Assignment 2}							% Minor Quiz 1, Major Quiz 2, End sem, etc
					
\begin{tabular}{cl}
1. & This question paper contains a total of 14 pages (14 sides of paper). Please verify.\\
2. & Write your name, roll number, department on \textbf{every side of every sheet} of this booklet\\
%3. & Write final answers \textbf{neatly with a pen} in the given boxes.\\
%4. & Do not give derivations/elaborate steps unless the question specifically asks you to provide these.
\end{tabular}


\begin{problem}{\textbf{SAT-based motion planning}} (20 points)

\bigskip
%\noindent
%In this assignment, we will explore how we can solve a simple motion planning problem using a SAT solver.

Let us assume that a 2D workspace
is divided into small rectangular blocks using a grid. The size of the workspace is $5 \times 5$. The lower left grid block has the ID $(0,0)$, and 
the upper right grid block has the ID $(4,4)$. The blocks $(2,0)$, $(3,0)$, $(1,2)$, $(3,2)$, $(1,4)$, and $(2,4)$ are covered with obstacles. 
We have two robots whose initial locations are $(0,0)$ and $(4,4)$, respectively. The robots have to move to the blocks $(4,4)$ and $(0,0)$, respectively.
The robots have four motion primitives: L, R, U, D that can take the robot from its current block location to the left, right, upper and lower block respectively.  
Each robot has a motion primitive that keeps it in the same grid block for one time unit. 


Write a SAT query whose solution will give the trajectories of the robots as the sequences of their motion primitives.
Generate the trajectories using MiniSAT solver. To find the solution iteratively, you should write a script to generate the Boolean formulas for a specific length 
of the trajectory automatically.


\bigskip
\noindent
Submit the following:
\begin{itemize}
\item The SAT formula for which you get a satisfiable solution.
\item A snapshot of the terminal showing the execution of the SAT solver.
\item Provide a visual representation of the trajectories synthesized from the solution obtained from the SAT solver.
\end{itemize}

\bigskip
\noindent
MiniSAT webpage:
\url{http://minisat.se}

\newpage
\ \\
\begin{minipage}{1\textwidth}
		\rectangle{\linewidth}{24cm}
		% \ruledrectangle{7}
\end{minipage}
\newpage
\ \\
\begin{minipage}{1\textwidth}
		\rectangle{\linewidth}{24cm}
		% \ruledrectangle{7}
\end{minipage}
\newpage
\ \\
\begin{minipage}{1\textwidth}
		\rectangle{\linewidth}{24cm}
		% \ruledrectangle{7}
\end{minipage}
\newpage
\ \\
\begin{minipage}{1\textwidth}
		\rectangle{\linewidth}{24cm}
		% \ruledrectangle{7}
\end{minipage}
\newpage
\ \\
\begin{minipage}{1\textwidth}
		\rectangle{\linewidth}{24cm}
		% \ruledrectangle{7}
\end{minipage}
\newpage
\ \\
\begin{minipage}{1\textwidth}
		\rectangle{\linewidth}{24cm}
		% \ruledrectangle{7}
\end{minipage}
\end{problem}


\newpage 
 
\begin{problem} {\textbf{SMT-based optimal LTL motion planning}} (20 points)

%In this assignment, we will explore how we can solve an LTL  motion planning problem using a SMT solver.
\bigskip
Let us assume that a 2D workspace is divided into small rectangular blocks using a grid. 
The size of the workspace is $5 \times 5$. The lower left grid block has the ID $(0,0)$, and 
the upper right grid block has the ID $(4,4)$. The blocks $(2,0)$, $(3,0)$, $(1,2)$, $(3,2)$, $(1,4)$, and $(2,4)$ are covered with obstacles.
The robots have motion primitives to perform one-step vertical, horizontal and diagonal movement. A vertical movement and a horizontal movement incur a cost of 1 unit, and a diagonal movement incurs a cost of $1.5$. Each robot has a motion primitive that keeps it in the same grid block for one time unit. 
This primitive incurs a cost of $0.5$.

Consider the following requirement. The robots have to explore the whole workspace and reach their final destinations. 
For a successful exploration, each obstacle-free grid cell should be visited by at least one robot for at least once. Suppose
that the initial location of the robots are $(0,0)$, $(0,1)$, $(1,0)$ and $(1,1)$ and the final location of the robots are $(0,0)$, $(0,4)$, $(4,0)$ and $(4,4)$.
The final locations are not assigned to individual robots. Any robot can go to any of the final locations.
Provide an LTL formula that captures the requirement stated above. 
A plan for this multi-robot system will be optimal if the total cost for all the movements involved in the plan is minimal.
Formulate the optimal plan generation problem as an SMT-solving problem and generate a motion plan using Z3 SMT solver.
To find the solution iteratively, you should write a script to generate the Boolean formulas for a specific length 
of the trajectory automatically.


\bigskip
\noindent
Submit the following:
\begin{itemize}
\item The LTL formula.
\item The SMT formula for which you get a satisfiable solution.
\item A snapshot of the terminal showing the execution of the SMT solver.
\item Provide a visual representation of the trajectories synthesized from the solution obtained from the SMT solver.
\end{itemize}

\bigskip
\noindent 
Z3 Wbdpage:
\url{https://rise4fun.com/Z3/tutorial/guide}

\newpage
\ \\
\begin{minipage}{1\textwidth}
		\rectangle{\linewidth}{24cm}
		% \ruledrectangle{7}
\end{minipage}
\newpage
\ \\
\begin{minipage}{1\textwidth}
		\rectangle{\linewidth}{24cm}
		% \ruledrectangle{7}
\end{minipage}
\newpage
\ \\
\begin{minipage}{1\textwidth}
		\rectangle{\linewidth}{24cm}
		% \ruledrectangle{7}
\end{minipage}
\newpage
\ \\
\begin{minipage}{1\textwidth}
		\rectangle{\linewidth}{24cm}
		% \ruledrectangle{7}
\end{minipage}
\newpage
\ \\
\begin{minipage}{1\textwidth}
		\rectangle{\linewidth}{24cm}
		% \ruledrectangle{7}
\end{minipage}
\newpage
\ \\
\begin{minipage}{1\textwidth}
		\rectangle{\linewidth}{24cm}
		% \ruledrectangle{7}
\end{minipage}
\end{problem}
\end{document}